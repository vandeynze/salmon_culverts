\documentclass[12pt]{elsarticle}
\usepackage{newtxtext}
\usepackage[margin=1in]{geometry}
%\pagestyle{empty}
\usepackage{natbib}
\bibliographystyle{unsrt}
\usepackage{color}
\usepackage{amsmath}
\usepackage{hyperref}
\usepackage{wrapfig}
\usepackage{titlesec}
\usepackage{colortbl}
\titleformat{\section}[runin]{\normalfont\bfseries}{\thesection.}{3pt}{}


\begin{document}

\begin{center} \textbf{BUDGET JUSTIFICATION} \end{center}

\noindent \textbf{JUSTIFICATION YEAR 1}\\

A. SALARIES AND WAGES\\

1. SENIOR PERSONNEL\\

a. Principal Investigator - \$29,436 Total\\

Assistant Professor, S.L.\ Jardine - 2.1 months (1.0 FTE) grantee: \$29,436. Jardine will oversee and manage the project, coordinate between the research team and scientific advisory board, mentor the postdoctoral scholar, Sea Grant Fellow and the SMEA Research Assistant, and provide technical assistance with optimization in R and developing the R Shiny app, and lead organizing outreach activities.\\

b. Co - Principal Investigator - \$62,400 Total\\

Postdoctoral Scholar and co Principal Investigator, B.\ Van Deynze - 12 months (1.0 FTE) Sea Grant: \$62,400. Van Deynze will co-administer the project, mentor the Sea Grant Fellow and the SMEA Research Assistant, assist with developing and deploying the optimization algorithm and R Shiny app, and assist with organizing outreach activities.\\

2. OTHER PERSONNEL\\

a. SMEA Research Assistant - \$11,291 Total\\
Research Assistant - 4.5 months (1.0 FTE) grantee: \$11,291. The Research Assistant will catalog existing methods for fish passage prioritization used in the Washington Case area and determine which methods can be expanded to include all barrier culverts in the Case Area.\\

b. Sea Grant Fellow -  \$7,527 Total\\
Sea Grant Fellow - 1.5 months (1.0 FTE) Sea Grant:  \$7,527. The Sea Grant Fellow will utilize the optimization framework to explore existing fish passage restoration plans and compare those plans to restoration plans that are identified by our framework as maximizing return on investment on fish passage in the Case Area. \\

B. FRINGE BENEFITS - \$28,801 Total\\

Sea Grant: \$19,562; grantee: \$9,239. Approved institutional rates were applied for each individual when calculating fringe benefits. Fringe benefits are calculated at a rate of 23.6\% for faculty, postdoctoral scholars, and research associates and 20.3\% for Sea Grant Fellows.\\

C. PERMANENT EQUIPMENT - \$0\\

D. EXPENDABLE SUPPLIES AND EQUIPMENT- \$1,700\\
Sea Grant: \$1,700. We have budgeted \$1,700 for a laptop computer for PI Van Deynze based on the cost of a MacBook pro with 16 GB of RAM and 512 GB of storage. Van Deynze's current postdoctoral position at the Northwest Fisheries Science Center and associated access to a laptop computer will end prior to the project start date and thus a laptop computer is needed for the entire project period.\\

E. TRAVEL - \$500\\
Sea Grant: \$500. We have budgeted \$500 for a one-day field trip to allow the project team to visit multiple barrier restoration projects throughout the Case Area representing high-, medium- and low-cost projects. Associated field trip costs include vehicle transportation costs (UCAR 8 passenger SUV rental rate of $\sim$ \$55), fuel cost (estimated at \$45), and per-diem meals and incidentals expenses for the PI, co-PIs, and two students (7 people * \$57.00 $\sim$ \$400).\\

F. PUBLICATION AND DOCUMENTATION COSTS - \$0\\

G. OTHER COSTS - \$8,545\\

a. Tuition for SMEA Research Assistant - \$5,545 Total\\
Grantee: \$5,545. The matching cost include $\sim$ 99\% of a quarter of tuition.\\

b. Tuition for the Sea Grant Fellow - \$5,614 Total\\
Sea Grant: \$5,614. The cost include one quarter of tuition.\\

c. Workshop costs - \$3,000 Total\\
Sea Grant: \$3,000. Workshop 1, held at the beginning of Year 1, is intended to uncover the objectives and challenges in culvert barrier replacement for key user groups and present our proposed framework and online tool to solicit feedback. The workshop will also assist our research team with cultivating a strong professional relationship within the community of stakeholders engaged in working on fish passage in the state of Washington. Workshop costs include event facility rental (estimated at \$440 based on the Walker Ames Room at the University of Washington campus), catering including breakfast, lunch, and snacks for $\sim$ 45 participants (estimated at \$34.50 per person * 45 participants = \$1552.5 based on an estimate from Arista Catering in Seattle), and the reimbursement of transportation costs for participants needing reimbursement (\$1,007.5 budgeted for transportation cost $\sim$ \$22 per person). \\

\noindent \textbf{JUSTIFICATION YEAR 2}\\

A. SALARIES AND WAGES\\

1. SENIOR PERSONNEL\\

a. Principal Investigator - \$30,176 Total\\

Assistant Professor, S.L.\ Jardine - 2.07 months (1.0 FTE) grantee: \$30,176. Jardine will oversee and manage the project, coordinate between the research team and scientific advisory board, mentor the postdoctoral scholar, Sea Grant Fellow and the SMEA Research Assistant, and provide technical assistance with optimization in R and developing the R Shiny app, and lead organizing outreach activities.\\

b. Co - Principal Investigator - \$32,448 Total\\

Postdoctoral Scholar and co Principal Investigator, B.\ Van Deynze - 6 months (1.0 FTE) Sea Grant: \$32,448. Van Deynze will co-administer the project, mentor the Sea Grant Fellow and the SMEA Research Assistant, assist with developing and deploying the optimization algorithm and R Shiny app, and assist with organizing outreach activities.\\


2. OTHER PERSONNEL\\

a. Sea Grant Fellow -  \$15,507 Total\\

Sea Grant Fellow - 1.5 months (1.0 FTE) Sea Grant:  \$15,507. The Sea Grant Fellow will utilize the optimization framework to explore existing fish passage restoration plans and compare those plans to restoration plans that are identified by our framework as maximizing return on investment on fish passage in the Case Area. \\


B. FRINGE BENEFITS - \$21,368 Total\\

Sea Grant: \$14,246; grantee: \$7,122. Approved institutional rates were applied for each individual when calculating fringe benefits. Fringe benefits are calculated at a rate of 23.6\% for faculty, postdoctoral scholars, and research associates and 20.3\% for Sea Grant Fellows.\\

C. PERMANENT EQUIPMENT - \$0\\

D. EXPENDABLE SUPPLIES AND EQUIPMENT- \$0\\

E. TRAVEL - \$0 \\

F. PUBLICATION AND DOCUMENTATION COSTS - \$3,000\\
Sea Grant: \$3,000. We budget \$3,000 in Year 2 to cover the costs associated with publishing our results in peer-reviewed literature and host our R Shiny app on the Shiny server for 1 year starting on the app launch date and near the end of Year 2 (pricing ranges from \$440-\$1,100 per year for the anticipated level of active hours logged for the app \url{https://www.shinyapps.io}).\\

G. OTHER COSTS - \$18,678 \\

a. Tuition for Sea Grant Fellow - \$11,678 \\
Sea Grant: \$11,678.\\

b. Workshop costs and video tutorial - \$7,000  Total\\
Sea Grant: \$7,000. Workshops 2 and 3 will be held in Year 2 of the project.  Workshop 2 will be an online workshop and Workshop 3 will be an in-person workshop. Workshop 2 is intended to present our preliminary findings and Workshop 3 is intended to launch our finalized online tool in an interactive session. The workshop will also assist our research team with cultivating a strong professional relationship within the community of stakeholders engaged in working on fish passage in the state of Washington. Workshop costs include event facility rental (estimated at \$440 based on the Walker Ames Room at the University of Washington campus), catering including breakfast, lunch, and snacks for $\sim$ 45 participants (estimated at \$34.50 per person * 45 participants = \$1552.5 based on an estimate from Arista Catering in Seattle), and the reimbursement of transportation costs for participants needing reimbursement (\$1,007.5 budgeted for transportation cost $\sim$ \$22 per person). The budget also includes the cost of video production to generate a tutorial for the online tool. Video production services will be contracted out, e.g.\ to the University of Washington Video Production team (estimated cost of \$4,000; see \url{https://www.washington.edu/video/our-services/production/}).  \\



\end{document}