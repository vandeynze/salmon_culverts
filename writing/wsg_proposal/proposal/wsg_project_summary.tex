\documentclass[12pt]{elsarticle}
\usepackage{newtxtext}
\usepackage[margin=1in]{geometry}
%\pagestyle{empty}
\usepackage{color}
\usepackage{amsmath}
\usepackage{hyperref}
\usepackage{wrapfig}
\usepackage{titlesec}
\usepackage{colortbl}
%\titleformat{\section}[runin]{\normalfont\bfseries}{\thesection.}{3pt}{}


\begin{document}
	
	\begin{center} \textbf{PROJECT SUMMARY} \end{center}
	
	%
		\section{Objectives} 
			% Briefly and clearly state the short-term and long-term objectives of the proposed work related to WSG goals and strategies. Incorporate both the scientific and societal purposes of the project in these objectives.
			\subsection*{Goals}
				\begin{enumerate}
					\item To contribute our expertise and knowledge in economics, optimization, and conservation biology to maximizing the returns on investment from barrier culvert restoration in Washington State.
					\item To learn about the diverse priorities, objectives, and concerns of various stakeholders as relates to barrier culvert restoration in Washington State.
					\item To train graduate students in producing high-quality, relevant, and reproducible applied research related to natural resources management.
				\end{enumerate}
			
			\subsection*{Objectives}
				\begin{enumerate}
					\item Catalog methods and datasets for fish passage prioritization indices for all counties, and any other entities using prioritization indices, within the Washington state injunction area. 
					\item Generate consistent predicted cost estimates for all barriers to fish passage within the Washington state injunction area. 
					\item Generate consistent habitat quality metrics associated with all barriers to fish passage within the Washington state injunction area for all five species of Pacific salmon and steelhead. 
					\item Develop a data-driven optimization framework for project prioritization, within the injunction area of Washington state, that synthesizes multiple geospatial datasets with statistical economic and ecological models, and incorporates stakeholder feedback based on multiple workshops, to identify restoration plans that maximize ecological, social, and economic objectives at a given funding level.
					\item Develop a customizable open-source online decision support tool (DST), along with a video tutorial, to make our optimization framework accessible to stakeholders, managers, and academics. 
				\end{enumerate}
			% Copy-paste from the narrative.
			% 1600 / 2000 character limit
			
		\section{Methodology} 
			% Succinctly describe the methods and approach to be used in accomplishing the objectives. 
			\begin{enumerate}
				\item We will catalog all methods and datsets used for fish passage prioritization within the Case Area. This catalog will allow for comparison across jurisdictions and serve as a starting point for considering data inputs into the optimization framework and DST.
				\item We will refine and apply predictive cost models currently being finalized by our team. We will generate consistent cost estimates for all inventoried barriers in the Case Area using statistical learning methods designed to maximize out-of-sample predictive power.
				\item We will use statistical models to relate measures of salmon habitat quality to features of the environment, building upon these earlier efforts to develop a consistent metric of salmon habitat quality based on instream temperature and flow data, as well as upland features related to riparian forest density and composition, road density, elevation, and watershed area.
				\item We will develop a novel optimization framework for identifying cost-effective restoration plans that meet multiple planning
				goals (e.g. salmon habitat gains, equity considerations, and risk mitigation) under a fixed budget, allowing for the identification of fish passage restoration plans associated with tradeoffs or "win-wins" across objectives.
				\item We will build and deploy a web-based DST that will allow resource managers and other users to explore and compare fish passage restoration plans generated by our optimization framework. Users will be able to compare optimal and user-selected plans in terms of objectives (e.g. habitat miles, equity, risk) and how relative performance is affected by different funding levels or levels of coordination between barrier owners.
				\item Throughout our project we will conduct a series of three workshops with resource managers and members of communities affected by restricted fish passage. Through these workshops we will refine our methods and DST to ensure that the resulting tool reflects managers' practical needs.
			\end{enumerate}
			% 1988 / 2000 character limit
		\section{Rationale}
			% Concisely state the problem or opportunity addressed. Indicate why the project is important, appropriate for WSG support, and why the proposed approach is necessary. Identify the expected outcomes of the project and potential project users.
			Across Western Washington, thousands of poorly-designed culverts at road crossings prevent migratory salmon from accessing potential habitat, hampering recovery efforts for declining populations. In 2013, a federal court found that barrier culverts on Washington state roads violate tribal treaty rights and issued an injunction requiring their replacement. While the injunction only applies to state-owned culverts, thousands of additional barrier culverts are owned by local governments and private landowners, often within the same watersheds leading to inter-dependencies between barrier correction activities. With growing funding and interest in restoring fish passage, there is an urgent need for consistent, science-based prioritization tools that account for the diverse objectives of barrier ownership entities and stakeholder groups i.e., those with treaty fishing rights, county and state resource managers.
			
			Our project will develop a consistent, data-driven framework for prioritizing fish passage barriers over multiple objectives. The decision support tool we develop will serve a coordinating function between barrier owners and stakeholders by allowing the consistent comparison of barrier correction plans. Users will be able to identify cost-effective plans that meet their specific user-defined objectives and budgets while considering opportunities to coordinate with other barrier owners, resulting in cost-effective plans that restore access to high-quality habitat as quickly as possible.
			
			Because barrier ownership is distributed across dozens of state, local, and private entities, resources for the development of methods and tools for assessing barrier correction plans consistently across the region is limited. Stakeholder engagement expertise and federal independence from state and local interests make Washington Sea Grant the ideal partner for this work.
			% 1900 / 2000 character limit
			
			
			
\end{document}