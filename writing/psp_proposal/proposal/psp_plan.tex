\begin{center} \textbf{SCIENCE WORK PLAN RELEVANCE AND BROADER IMPACTS} \end{center}

Our framework presents an alternative and complement to the ``rank and score'' methods, or prioritization indices (PI), currently used by Washington State and various other actors (e.g.\ counties) for prioritizing barrier culverts for restoration. These methods, frequently variations of WDFW's methods, are essentially a weighted sum of factors that drive the benefits, and sometimes costs, associated with correcting a single barrier culvert in isolation. Points for each factor are assigned based on whether specific metrics, measured via field survey or GIS tools, fall within specified ranges. For example, a PI may be increasing habitat quantity and quality metrics for all five species of salmon, decreasing in the number of barriers downstream, and decreasing in estimated project cost. 

Additionally, the fact that PI methods vary across ownership entities prevents consistent comparisons of the relative benefits, or costs, of barrier correction across regions and ownership. Our framework will incorporate all barrier culverts in Washington State, making such comparisons possible.