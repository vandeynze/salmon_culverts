\begin{center} \textbf{SCIENCE WORK PLAN RELEVANCE AND BROADER IMPACTS} \end{center}

%Our framework presents an alternative and complement to the ``rank and score'' methods, or prioritization indices (PI), currently used by Washington State and various other actors (e.g.\ counties) for prioritizing barrier culverts for restoration. These methods, frequently variations of WDFW's methods, are essentially a weighted sum of factors that drive the benefits, and sometimes costs, associated with correcting a single barrier culvert in isolation. Points for each factor are assigned based on whether specific metrics, measured via field survey or GIS tools, fall within specified ranges. For example, a PI may be increasing habitat quantity and quality metrics for all five species of salmon, decreasing in the number of barriers downstream, and decreasing in estimated project cost. 

%Additionally, the fact that PI methods vary across ownership entities prevents consistent comparisons of the relative benefits, or costs, of barrier correction across regions and ownership. Our framework will incorporate all barrier culverts in Washington State, making such comparisons possible.

Our project is most directly aligned with Priority Science Action 1 ``Evaluate how current and future social, economic, and political factors, such as population growth and urban development, will affect habitat quality and quantity, both negatively and positively as gauged by salmon viability.'' and Priority Science Action 12: ``Refine risk assessment tools and scenario development and analyses to improve our understanding of highly uncertain, complex and inter-related challenges and solutions to provide information that can be used to identify actions to achieve a more resilient Puget Sound ecosystem.'' 

Culvert barrier corrections are expected to have large impacts on salmon habitat and viability and will be shaped by political factors as well as future population growth. Our analysis will compare habitat outcomes generated by independent actors using ``rank and score'' methods to those from coordinated conservation planning based on principles of optimization. Additionally, coordinating culvert barrier correction across multiple agencies with distinct and limited budgets is inherently a complex challenge. Our project will develop a novel framework that will demonstrate trade-offs between potentially competing priorities shared with stakeholders via an online DST. Our approach is consistent with a number of recommendations set forth by the Science Panel, including incorporation of indigenous knowledge, coordination of efforts across entities, communication targeted to appropriate audiences, and the evaluation of trade-offs between approaches.