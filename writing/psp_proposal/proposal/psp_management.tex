\begin{center} \textbf{PROJECT MANAGEMENT} \end{center}

The proposed project has been co-developed by the project PIs and members of the FPPAC. The PIs and FPPAC will continue to collaborate to produce an integrated project leading to both actionable and publishable science. 

\textbf{PI Jardine} will serve as the lead administrator of the grant where administrative responsibilities include organizing meetings both internally and with the scientific advisory board to ensure the project is on track to deliver work products on time, tracking project performance, facilitating project design decisions, and supervising and mentoring the postdoctoral scholar and the two Research Assistants (one funded with PSP funds and the other with leveraged SMEA funds). PI Jardine will also provide technical assistance with optimization in R and developing the R Shiny app including providing template code for various optimization algorithms along with sensitivity analyses and model selection, and providing template code for the DST user app features.

\textbf{Co-PI Van Deynze} will co-administer the grant, with PI Jardine, where administrative responsibilities include organizing meetings both internally and with the scientific advisory board to ensure the project is on track to deliver work products on time, tracking project performance, facilitating project design decisions, and mentoring the two Research Assistants. Co-PI Van Deynze will also assist with developing and deploying the optimization algorithm and the R Shiny app. As a post-doctoral research associate, co-PI Van Deynze will receive training from PI Jardine in the development of applied optimization algorithms, in project administration, and in mentorship of students, furthering development as an independent researcher. 

\textbf{Co-PI Scheuerell} will co-supervise the Research Scientist and provide assistance for development of the habitat quality index and measures of risk with respect to salmon. He will also support the engagement and outreach effort, focusing on the benefits to salmon and stakeholders. Additionally, co-PI Scheuerell will liaise with state and tribal co-managers regarding ongoing salmon conservation and recovery efforts.


\textbf{Co-PIs Fonner and Holland} will provide assistance as needed for development and deployment of the multi-criteria optimization framework and corresponding algorithm. They will also support the engagement and outreach effort, facilitating incorporation of best practices and paradigms from conservation social sciences and provide guidance on the institutional environment and human dimensions associated with salmon conservation and recovery.


%\textbf{Students}: The  SMEA Research Assistant$^\dagger$ will catalog methods and datasets for fish passage prioritization indices for all counties, and any other entities using prioritization indices, within the Case Area. The RA will also determine the data availability of variables used in the prioritization index of each entity for all barrier culverts in the Washington State injunction area. Through their work, the RA will become deeply familiar with a diverse set of prioritization methods and the practical considerations of their design (e.g.\ resource, information, and interpretability constraints), preparing them for a professional role in resource policy or further graduate studies. The SMEA/QERM Research Assistant will utilize the optimization framework to explore the research questions identified in the Project Narrative as a basis for a thesis. The Research Scientist will develop the habitat quality index under the supervision of PI Scheuerell. Efforts include reviewing the relevant literature, gathering data, calculating the index, and documenting all analyses such that they are reproducible.\\
%

%The \textbf{Fish Passage Policy Advisory Committee (FPPAC} consists of Jeff Dickison (Squaxin Island Tribe), Steve Hinton (Tulalip Tribes); Marc Duboiski (Washington State Recreation and Conservation Office); Dave Caudill (Washington State Recreation and Conservation Office); and Evan Lewis (King County Fish Restoration Program). The FPPAC will guide the project and facilitate engagement with tribes, barrier owners, and other stakeholders.\\


%\noindent $^\dagger$The SMEA Research Assistant will be paid for with leveraged funds provided by SMEA.
