\begin{center} \textbf{PROJECT SUMMARY} \end{center}


%(250-word limit) to convey objectives, methodology, and rationale. This should provide a
%clear and concise description of the proposed work in terms that are understandable by individuals who are
%not experts in your field.

%Right now at 240 words.

			Across Western Washington, thousands of poorly-designed culverts at road crossings prevent migratory salmon from accessing potential habitat, hampering recovery efforts for declining populations. In 2013, a federal court found that barrier culverts in Washington violate tribal treaty rights and issued an injunction requiring their replacement. While the injunction only applies to state-owned culverts, thousands of additional barrier culverts are owned by local governments and private landowners, often within the same watersheds leading to inter-dependencies between barrier correction activities. 
			
			Counties and other actors are independently ramping up efforts to correct barriers to fish passage in their jurisdiction resulting in prioritization frameworks that only allow for comparison within a given entity's jurisdiction, complicating the coordination of investments in fish passage across Western Washington. Our project will develop a consistent, data-driven framework for prioritizing fish passage barriers over multiple objectives, drawing from a rich literature on fish passage restoration plans that maximize return on investment. 
			
			Our optimization framework will enable us to characterize the tradeoffs associated with barrier culvert replacement, estimate gains from coordination across barrier culvert owners in Washington, explore how the nature of funding streams impacts conservation outcomes associated with barrier culvert replacement, and investigate how path dependency affects optimal barrier culvert restoration strategies. The decision support tool we develop will serve a coordinating function between barrier owners and managers by allowing the consistent comparison of alternative barrier correction, complementing ongoing efforts to restore fish passage in the state of Washington.
