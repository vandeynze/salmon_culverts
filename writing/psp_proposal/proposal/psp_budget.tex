\begin{center} \textbf{BUDGET \& BUDGET JUSTIFICATION} \end{center}


\section{Project YR 1 (10/1/21-6/30/22)}
\begin{enumerate}
\item Salary Cost
\begin{itemize}
\item  \$15,122 is requested for 1 month of effort from PI Jardine. Jardine will serve as the lead administrator of the grant where administrative responsibilities include organizing meetings both internally and with the scientific advisory board to ensure the project is on track to deliver work products on time, tracking project performance, facilitating project design decisions, and supervising and mentoring the postdoctoral scholar and the two Research Assistants (one funded with PSP funds and the other with leveraged SMEA funds). PI Jardine will also provide technical assistance with optimization in R and developing the R Shiny app including providing template code for various optimization algorithms along with sensitivity analyses and model selection, and providing template code for the DST user app features.
\item \$55,800 is requested for 9 months of effort from co-PI and postdoctoral scholar VanDeynze. VanDeynze will co-administer the grant, with PI Jardine, where administrative responsibilities include organizing meetings both internally and with the scientific advisory board to ensure the project is on track to deliver work products on time, tracking project performance, facilitating project design decisions, and mentoring the two Research Assistants. Co-PI Van Deynze will also assist with developing and deploying the optimization algorithm and the R Shiny app. As a post-doctoral research associate, co-PI Van Deynze will receive training from PI Jardine in the development of applied optimization algorithms, in project administration, and in mentorship of students, furthering development as an independent researcher. 
\item \$38,925 is requested for 9 months of effort from a post-graduate Research Scientist. The Research Scientist will develop the habitat quality index under the supervision of co-PI Scheuerell. Efforts include reviewing the relevant literature, gathering data, calculating the index for all barrier culverts in the Washington Case Area, and documenting all analyses such that they are reproducible. Additionally, the Research Scientist will collaborate with PI Jardine and co-PI VanDeynze to publish a streamlined version of the R shiny app. Thus, we will recruit a Research Scientist with a strong background in fisheries sciences and strong quantitative skills.
\item \$22,590 is requested for 4.5 months of effort of a graduate student Research Assistant. The Research Assistant will utilize the optimization framework to explore the research questions identified in the Project Narrative as a basis for a thesis, exploring the research questions laid out in the proposal. In the first year, prior to the optimization framework being finalized, the Research Assistant will familiarize themselves with all aspects of the project including the relevant literature, the data, optimization algorithms, and co-develop code to analyze preliminary results based on maximizing habitat quantity subject to a budget constraint and potential coordination constraints (e.g.\ budgets must be spent entirely in a single county). The code will serve as a template for analyzing results once other dimensions of the problem are incorporated. Thus we will recruit a Research Assistant with interdisciplinary training or interests in political economy, economics, and fisheries as well as someone with strong quantitative skills and coding experience.
\end{itemize}
\item Benefits
\begin{itemize}
\item Fringe benefits are charged at the standard UW rates of 23.6\% for faculty and postdoctoral scholars, 21.6\% for professional employees, and 21.6\% for graduate student research assistants. Total fringe benefits are equal to \$32,777.
\end{itemize}
\item Overhead
\begin{itemize}
\item Indirect costs, for on-campus research projects, are 54.5\% of salaries, benefits, expendable supplies and equipment, travel, and publication costs. Total indirect costs are \$94,746.
\end{itemize}
\item Supplies
\begin{itemize}
\item We have budgeted \$2,000 for a laptop computer for PI Van Deynze based on the cost (Education Pricing) of a MacBook pro with 16 GB of RAM and 1 TB of storage. Van Deynze's current postdoctoral position at the Northwest Fisheries Science Center and associated access to a laptop computer will end prior to the project start date and thus a laptop computer is needed for the entire project period.
\end{itemize}
\item Travel
\begin{itemize}
\item We have budgeted \$500 for a one-day field trip to allow the project team to visit multiple barrier restoration projects throughout theCase Area representing high-, medium- and low-cost projects. Associated field trip costs include vehicle transportation costs, fuel cost (estimated at \$45), and per-diem meals and incidentals expenses for the PI, co-PIs, and two students (7 people * \$57.00 $\sim$ \$400).
\end{itemize}
\item Other direct costs
\begin{itemize}
\item We have budgeted \$16,842 for tuition for the Research Assistant (see description of the Research Assistant's role in the itemized Salary section).
\item We have budgeted \$3,000 for Workshop 1, held at the beginning of Year 1, is intended to uncover the objectives and challenges in culvert barrier replacement for key user groups and present our proposed framework and online tool to solicit feedback. The workshop will also assist our research team with cultivating a strong professional relationship within the community of stakeholders engaged in working on fish passage in the state of Washington. Workshop costs include event facility rental (estimated at \$440 based on the Walker Ames Room at the University of Washington campus), catering including breakfast, lunch, and snacks for ? 45 participants (estimated at \$34.50 per person * 45 participants = \$1552.5 based on an estimate from Arista Catering in Seattle), and the reimbursement of transportation costs for participants needing reimbursement (\$1,007.5 budgeted for transportation cost at \$22 per person).
\end{itemize}
\end{enumerate}


\section{Project YR 2 (7/1/22-6/30/23)}
\begin{enumerate}
\item Salary Cost
\begin{itemize}
\item  \$15,727 is requested for 1 month of effort from PI Jardine. Jardine will serve as the lead administrator of the grant where administrative responsibilities include organizing meetings both internally and with the scientific advisory board to ensure the project is on track to deliver work products on time, tracking project performance, facilitating project design decisions, and supervising and mentoring the postdoctoral scholar and the two Research Assistants (one funded with PSP funds and the other with leveraged SMEA funds). PI Jardine will also provide technical assistance with optimization in R and developing the R Shiny app including providing template code for various optimization algorithms along with sensitivity analyses and model selection, and providing template code for the DST user app features.
\item \$77,376 is requested for 12 months of effort from co-PI and postdoctoral scholar VanDeynze. VanDeynze will co-administer the grant, with PI Jardine, where administrative responsibilities include organizing meetings both internally and with the scientific advisory board to ensure the project is on track to deliver work products on time, tracking project performance, facilitating project design decisions, and mentoring the two Research Assistants. Co-PI Van Deynze will also assist with developing and deploying the optimization algorithm and the R Shiny app. As a post-doctoral research associate, co-PI Van Deynze will receive training from PI Jardine in the development of applied optimization algorithms, in project administration, and in mentorship of students, furthering development as an independent researcher. 
\item \$26,988 is requested for 6 months of effort from a post-graduate Research Scientist. The Research Scientist will collaborate with PI Jardine and co-PI VanDeynze to publish and deploy a finalized version of the R shiny app.
\item \$23,495 is requested for 4.5 months of effort of a graduate student Research Assistant. The Research Assistant will utilize the optimization framework to explore the research questions identified in the Project Narrative as a basis for a thesis, exploring the research questions laid out in the proposal.
\end{itemize}
\item Benefits
\begin{itemize}
\item Fringe benefits are charged at the standard UW rates of 23.6\% for faculty and postdoctoral scholars, 21.6\% for professional employees, and 21.6\% for graduate student research assistants. Total fringe benefits are equal to \$34,609.
\end{itemize}
\item Overhead
\begin{itemize}
\item Indirect costs, for on-campus research projects, are 54.5\% of salaries, benefits, expendable supplies and equipment, travel, and publication costs. Total indirect costs are \$106,113.
\end{itemize}
\item Supplies
\begin{itemize}
\item None
\end{itemize}
\item Travel
\begin{itemize}
\item We have budgeted \$3,000 for PI Jardine and the Research Assistant to travel to one domestic academic conference. For example, the AERE Annual Summer Conference, which is held in June each year. The 2023 location has not yet been announced.
\end{itemize}
\item Other direct costs
\begin{itemize}
\item We budget \$3,000 in Year 2 to cover the costs associated with publishing our results in peer-reviewed literature and host our R Shiny app on the Shiny server for 1 year starting on the app launch date and near the end of Year 2 (pricing ranges from \$440-\$1,100 per year for the anticipated level of active hours logged for the app \url{https://www.shinyapps.io}). Note at the time of writing this proposal, we are actively exploring options for long-term hosting of our app to ensure longevity in our research impact. One possibility is the NOAA toolbox \url{https://noaa-fisheries-integrated-toolbox.github.io/}.
\item We have budgeted \$17,346 for tuition for the Research Assistant (see description of the Research Assistant's role in the itemized Salary section).
\item We have budgeted \$3,000 for workshop 2 and 3 and a video tutorial. Workshops 2 and 3 will be held in Year 2 of the project. Workshop 2 will be an online workshop and Workshop 3 will be an in-person workshop. Workshop 2 is intended to present our preliminary findings and Workshop 3 is intended to launch our finalized online tool in an interactive session. The workshop will also assist our research team with cultivating a strong professional relationships with tribes and within the community of stakeholders engaged in working on fish passage in the state of Washington. Workshop costs include event facility rental (estimated at \$440 based on the Walker Ames Room at the University of Washington campus), catering including breakfast, lunch, and snacks for $\sim$ 45 participants (estimated at \$34.50 per person * 45 participants = \$1552.5 based on an estimate from Arista Catering in Seattle), and the reimbursement of transportation costs for participants needing reimbursement (\$1,007.5 budgeted for transportation cost $\sim$ \$22 per person). The budget also includes the cost of video production to generate a tutorial for the online tool. Video production services will be contracted out, e.g. to the University of Washington Video Production team (estimated cost of \$4,000; see \url{https://www.washington.edu/video/our-services/production/}).
\end{itemize}
\end{enumerate}