\documentclass[12pt]{elsarticle}
\usepackage{newtxtext}
\usepackage[margin=1in]{geometry}
%\pagestyle{empty}
\usepackage{color}
\usepackage{amsmath}
\usepackage{hyperref}
\usepackage{wrapfig}
\usepackage{titlesec}
\usepackage{colortbl}
%\titleformat{\section}[runin]{\normalfont\bfseries}{\thesection.}{3pt}{}


\begin{document}
	
	\begin{center} \textbf{Washington Salmon Restoration Research Assistantship} \end{center}

\section*{Description}

			Across Western Washington, thousands of poorly-designed culverts at road crossings prevent migratory salmon from accessing potential habitat, hampering recovery efforts for declining populations. In 2013, a federal court found that barrier culverts in Washington violate tribal treaty rights and issued an injunction requiring their replacement. While the injunction only applies to state-owned culverts, thousands of additional barrier culverts are owned by local governments and private landowners, often within the same watersheds leading to inter-dependencies between barrier correction activities. 
			
			Counties and other actors are independently ramping up efforts to correct barriers to fish passage in their jurisdictions relying on prioritization index scoring, an approach that only allows for comparison within a given entity's jurisdiction, complicating the coordination of investments in fish passage across Western Washington. The Research Assistant (RA) will explore differences and similarities in prioritizing barriers for fish passage removal across Western Washington. Specifically, the RA will:
			
			\begin{enumerate}
			\item Catalog the factors (variables) included in prioritization index (PI) scores across all relevant entities in Western Washington
			\item Create a geospatial dataset of all barriers in Western Washington with a PI score 
			\item Determine the degree to which various PI methods used can be used to categorize all barriers to fish passage in Western Washington, e.g.\ determine whether there exist the necessary data to apply PI methods from King County to barriers outside of King County
			\item Draft a manuscript discussing the differences and similarities in PI methods across Western Washington and the implications of these differences/similarities for salmon conservation
			\item Assist with hosting workshops for resource managers and tribal representatives to understand objectives and concerns around restoring fish passage in Western Washington
			\end{enumerate}

\section*{Supervisors and Collaborators}
			
The RA position will be jointly supervised by Sunny Jardine and Braeden VanDeynze at the University of Washington (UW) in the School of Marine and Environmental Affairs and have the opportunity to collaborate with Dan Holland and Robby Fonner at the Northwest Fisheries Science Center and Mark Scheuerell of the United States Geological Survey (USGS) and UW's School of Aquatic and Fishery Sciences.


\section*{Qualifications and Requirements}
There are no qualifications needed to be successful in this position. If the applicant does not have experience working with and generating geospatial data, they must agree to taking Jardine's course, SMEA 550 Spatial Data workshop, in the spring of 2022.
			
\section*{Contact}
\noindent Please reach out to Sunny Jardine (jardine\@uw.edu) with any questions regarding the position.
			
\end{document}