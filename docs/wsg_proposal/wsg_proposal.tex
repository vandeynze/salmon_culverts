\documentclass[12pt]{elsarticle}
\usepackage{times}
\usepackage{amsmath}
\usepackage{amssymb}
\usepackage{natbib}
\bibliographystyle{unsrt}
\usepackage{hyperref}
\usepackage{setspace}
\usepackage[section]{placeins}
\usepackage{graphicx}
\usepackage{mathrsfs}
\usepackage[margin=1in]{geometry}
\pagestyle{empty}
\usepackage{titlesec}
\titleformat{\section}[runin]{\normalfont\bfseries}{\thesection.}{3pt}{}


\begin{document}

\begin{center} \textbf{PROJECT NARRATIVE} \end{center}

%
\section{Goals and objectives}


%
\section{Background}
\subsection{Fish passage in Washington state}
\subsection{Optimization tools in fish passage}
\subsection{Optimization for Washington fish passage}

%
\section{Approach}
\subsection{Cost model}
\subsection{Habitat model}
\subsection{Equity}
\subsection{Risk mitigation}

\subsection{Network configuration}
\subsection{Ownership}
\subsection{Optimization}

We will explore two alternative approaches to define cost-effective restoration plans that meet multiple planning goals (e.g.\ salmon habitat gains, equity considerations, and risk mitigation). Both approaches are considered ``a priori'' methods for multiobjective optimization because the decision maker must specify their preferences related to the various objectives prior to the optimization. 

In the first approach, called linear scalarization, managers specify weights for each of the objectives and the weighted sum of the objectives is minimized. In the second approach, called the $\epsilon-$constraint method, one objective function is minimized and upper bounds, $\epsilon$ parameters are provided for all remaining objectives.




\subsection{Role of team members and partners}
\textbf{PI Jardine} will serve as the lead administrator of the grant where administrative responsibilities include organizing meetings both internally and with the scientific advisory board to ensure the project is on track to deliver work products on time, tracking project performance, facilitating project design decisions, and supervising and mentoring the postdoctoral scholar and the Sea Grant Fellow.  PI Jardine will also provide technical assistance with optimization in R and developing the R Shiny app including providing template code for various optimization algorithms along with sensitivity analyses and model selection, and providing template code for the DST user app features.\\


%
\section{Engagement plan}
\subsection{Community collaborators} 
\subsection{Target audiences}
\subsection{Engagement activities}
\subsection{Anticipated outcomes and evaluation}

\clearpage
\large References\\
\normalsize
\bibliography{wsg}
\end{document}